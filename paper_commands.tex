
\usepackage{xcolor}
\hypersetup{
    colorlinks=true,
    linkcolor=tsutsuji,
    citecolor=kikyou,
    urlcolor=kon-peki
}
\definecolor{ama-iro}{RGB}{0, 158, 243.0} % Light blue, can be used for headers or highlights
\definecolor{fuyu-gaki}{HTML}{C75146} % Persimmon red, ideal for warnings or critical points
\definecolor{momiji}{RGB}{245, 70, 111} % Bright red, great for emphatic notes or alerts
\definecolor{hotaru-bi}{RGB}{229,221,58} % Yellowish, for tips or hints
\definecolor{kon-peki}{RGB}{1,120,217} % Deep blue, suitable for section titles
\definecolor{shin-kai}{RGB}{77,98,152} % Steel blue, good for subtle text or links
\definecolor{shin-ryoku}{RGB}{1,145,97} % Green, use for success messages or "OK" status
\definecolor{yama-budo}{RGB}{171,14,122} % Deep magenta, excellent for key points or notes
\definecolor{tsutsuji}{HTML}{C71585} % Magenta-pink, use for "prove or disprove" statements
\definecolor{mizu}{rgb}{0.0, 0.6, 0.8} % Water (light blue), good for background shading
\definecolor{kikyou}{rgb}{0.4, 0.4, 0.8} % Bellflower (blueish purple), appropriate for emphasis or subdued headers
\definecolor{keshizumi}{RGB}{110, 110, 110} % Ash gray, suitable for neutral
\definecolor{usuzumi}{RGB}{140, 140, 140} % Light ink gray,

\definecolor{theoremcolor}{RGB}{229,221,58}
\definecolor{definitioncolor}{RGB}{255, 213, 179}
\definecolor{examplecolor}{RGB}{255, 213, 179}


\newcommand{\highlight}[1]{\textcolor{shin-ryoku}{#1}}
\newcommand{\ari}[1]{\textcolor{momiji}{\, \textbf{Ari(Comment):}}\, \textcolor{shin-ryoku}{#1}}
\newcommand{\markComment}[1]{\textcolor{orange}{\textbf{Mark (Comment):}}\textcolor{blue}{\quad#1}}

\newcommand{\clement}[1]{\textcolor{purple}{\textbf{Cl\'ement(Comment):}}\textcolor{ama-iro}{\quad#1}}
\newcommand{\ccnote}[1]{{{\footnote{\color{ama-iro}\textbf{Cl\'ement(Comment):}\quad#1}}}}

\newcommand{\sstext}[1]{\textcolor{kikyou}{\textbf{Satchit(Comment):}}\textcolor{yama-budo}{\quad#1}}
\newcommand{\ssnote}[1]{{{\footnote{\textcolor{yama-budo}{\textbf{Satchit(Comment):}\quad#1}}}}}




% Common stuff that show up in nearly all writeups
\newcommand{\Def}{\overset{\mathtt{def}}{=}}

\makeatletter
\newcommand{\smalldollar}{\mathrel{\mathpalette\small@dollar\relax}}
\newcommand{\small@dollar}[2]{%
  \vcenter{\hbox{%
    $#1\textnormal{\fontsize{0.7\dimexpr\f@size pt}{0}\selectfont\$}$%
  }}%
}
\makeatother

% General
%\renewcommand{\emph}[1]{\textcolor{kon-peki}{\textit{#1}}}
\newcommand{\Resolved}{\textcolor{fuyu-gaki}{\textbf{Resolved }}}
\newcommand{\TODO}{\textcolor{orange}{??TODO??}}
\newcommand{\Alg}{\mathsf{Alg}}
\newcommand{\Set}[1]{\left\{ #1\right\}}
\newcommand{\Size}[1]{\left| #1 \right|}
\newcommand{\RCloseLOpenInterval}[2]{}
\newcommand{\LCloseROpenInterval}[2]{\left[#1, #2\right)}
\newcommand{\Interval}[2]{\left[\right]}
\newcommand{\OpenInterval}[2]{\left(\right)}

\newcommand{\leftarrowS}{\leftarrow\joinrel\smalldollar}
\newcommand{\xleftarrowS}[1]{\overset{#1}{\leftarrowS}}
\newcommand{\rightarrowS}{\smalldollar\joinrel\rightarrow}
\newcommand{\negl}{\mathtt{negl}}
\newcommand{\Indicator}[1]{\mathbf{1}\left[#1\right]}
\newcommand{\Eps}{\varepsilon}
\renewcommand{\tilde}[1]{\widetilde{#1}}
\newcommand{\bit}{\{0,1\}}
\newcommand{\bitStrings}{\bit^*}
\newcommand{\poly}{\operatorname{poly}}
\newcommand{\Naturals}{\mathbb{N}}
\newcommand{\Reals}{\mathbb{R}}
\newcommand{\BigO}[1]{O\left(#1\right)}
\newcommand{\BigOTilde}[1]{\tilde{O}\left(#1\right)}
\newcommand{\SmallO}[1]{o\left(#1\right)}
\newcommand{\BigOmega}[1]{\Omega\left(#1\right)}
\newcommand{\BigOmegaTilde}[1]{\tilde{\Omega}\left(#1\right)}
\newcommand{\BigTheta}[1]{\Theta\left(#1\right)}
\newcommand{\BigThetaTilde}[1]{\tilde{\Theta}\left(#1\right)}
\newcommand{\SmallOmega}[1]{\omega\left(#1\right)}
\newcommand{\FuncDomain}{\mathcal{X}}
\newcommand{\FuncRange}{\mathcal{Y}}
\newcommand{\HashFamily}{\mathcal{H}}
\newcommand{\Loss}{\mathcal{L}}
\newcommand{\Oracle}{\mathcal{O}}


% Complexity classes 
\newcommand{\BPP}{\mathsf{BPP}}
\newcommand{\NP}{\mathsf{NP}}
\newcommand{\HVSZK}{\mathsf{HVSZK}}
\newcommand{\ED}{\mathsf{ED}}
\newcommand{\SD}{\mathsf{SD}}
\newcommand{\Yes}{\mathsf{Yes}}
\newcommand{\No}{\mathsf{No}}
\newcommand{\SampleComplexityFunc}{s}
\newcommand{\SampleComplexity}[1]{s(#1)}
\newcommand{\CommComplexity}[1]{c(#1)}
\newcommand{\RoundComplexity}[1]{r(#1)}

% Probability Distributions
\newcommand{\RV}[1]{\textcolor{tsutsuji}{#1}}
\newcommand{\TV}[2]{d_{\text{TV}}\left(#1,#2\right)}
\newcommand{\RL}[2]{\texttt{RL}\left(#1,#2\right)}
\newcommand{\EMD}[2]{\mathrm{EMD}\left(#1,#2\right)}
\newcommand{\Property}{\Pi}
\newcommand{\Domain}{[N]}
\newcommand{\DomainSize}{N}
\newcommand{\samples}{\leftarrowS} % See top of doc for special symb
\newcommand{\iidSamples}{\xleftarrowS{\text{i.i.d}}}
\newcommand{\uniformSamples}{\xleftarrowS{\text{uniform}}}
\newcommand{\Dist}{\mathcal{D}}
\newcommand{\DistPrime}{\textcolor{black}{\mathcal{D}'}}
\newcommand{\Ent}{\mathsf{H}}
\newcommand{\Uniform}[1]{\mathsf{Uniform}\left[#1\right]}
\newcommand{\DistSet}[1]{\Delta\left(#1\right)}
\newcommand{\Support}[1]{\text{Supp}\left({#1}\right)}
\newcommand{\SupportSize}[1]{\left|\Support{#1}\right|}
\newcommand{\Mean}[2]{\underset{#1}{\mathbb{E}}\left[#2\right]}
\newcommand{\Prob}[1]{\Pr\left[#1 \right]}
\newcommand{\PProb}[2]{\Pr_{#2}\left[#1 \right]}
% \newcommand{\TV}[1]{d_{\mathsf{TV}}\left(#1\right)}
\newcommand{\RenyiEnt}[1]{\mathsf{H}_{#1}}
\newcommand{\MinEnt}{\RenyiEnt{\infty}}
\newcommand{\ProductDist}[1]{{\overset{#1}\otimes}\Dist}
\newcommand{\weight}[1]{\mathsf{wt}\left(#1\right)}
\newcommand{\SamplingOracle}{\Dist}
\newcommand{\PMFOracle}{\Dist_{\textcolor{blue}{\texttt{pmf}}}}

% Learning, Truth and Claim
\newcommand{\Claimed}[1]{\textcolor{usuzumi}{\widetilde{#1}}}
\newcommand{\ClaimedDist}[1]{\Claimed{\Dist}\left[#1\right]}
\newcommand{\Tag}[1]{\Claimed{\BucketOf{\Proximity}{#1}}}
\newcommand{\True}[1]{\textcolor{black}{#1}}
\newcommand{\TrueDist}[1]{{\True{\Dist}\left[#1\right]}}
\newcommand{\Learned}[1]{\textcolor{kon-peki}{\overline{#1}}}
\newcommand{\LearnedDist}[1]{{\Learned{\Dist}\left[#1\right]}}

\newcommand{\EmpDist}[2]{\textcolor{shin-ryoku}{\overline{#1}}_{#2}}

% Histograms
\newcommand{\NumBuckets}{L_\Proximity}
\newcommand{\Bucket}[2]{B^{(#1)}_{\Proximity}\left(#2\right)}
\newcommand{\BucketOf}[2]{\text{Bucket}_{#1}(#2)}
\newcommand{\HistFactor}{\tau'}
\newcommand{\TrueHist}{\{\True{p_j}\}_{j \in \BucketIndices}}
\newcommand{\EmpHist}{\{\Learned{\overline{p}_j}\}_{j \in \BucketIndices}}
\newcommand{\ClaimedHist}[3]{\textcolor{fuyu-gaki}{\widetilde{\Hist{#1}{#2}}}_{#3}}
\newcommand{\ClaimHistExpanded}{\{\Claimed{p_j}\}_{j \in \BucketIndices}}
\newcommand{\BucketIndices}{[\NumBuckets]}
\newcommand{\LoIndex}{L}
\newcommand{\HiIndex}{\lfloor \frac{\log\DomainSize}{\HistFactor}\rfloor}

\newcommand{\ApproxHistParam}{(\DomainSize, \Proximity)}
\newcommand{\DistsToHist}[1]{\mathsf{Hist}^{\ApproxHistParam}(#1)}
\newcommand{\HistMapsTo}[1]{\mathcal{F}^{\ApproxHistParam}(#1)}
\newcommand{\ApproxHistTruth}{\{\textcolor{cadmiumgreen}{p}_j\}_{j \in \BucketIndices}}
\newcommand{\ApproxHistAlleged}{\{\textcolor{red}{\tilde{p}}_j\}_{j \in \BucketIndices}}
\newcommand{\ApproxHistEmpirical}{\{\textcolor{blue}{\widehat{p}}_j\}_{j \in \BucketIndices}}
\newcommand{\Mass}[1]{\textcolor{cadmiumgreen}{p}_{#1}}
\newcommand{\MassEmpirical}[1]{\textcolor{blue}{\widehat{p}}_{#1}}
\newcommand{\MassAlleged}[1]{\textcolor{red}{\tilde{p}}_{#1}}
\newcommand{\MassUpper}[2]{\frac{e^{(#1+1)#2}}{\DomainSize}}
\newcommand{\MassUpperShort}[1]{\frac{e^{(#1+1)\HistFactor}}{\DomainSize}}
\newcommand{\MassLower}[2]{\frac{e^{#1\HistFactor}}{\DomainSize}}
\newcommand{\MassLowerShort}[1]{\frac{e^{#1\HistFactor}}{\DomainSize}}


% Entropy Gap Protocol
\newcommand{\gap}{\nu}
\newcommand{\FlatnessParam}{\beta}

% Interactive Proofs
\newcommand{\Proof}{\pi}
\newcommand{\Transcript}{\mathsf{Tr}}
\newcommand{\Reject}{\textcolor{black}{\mathsf{Reject}}}
\newcommand{\Fail}{\textcolor{black}{\mathsf{FAIL}}}
\newcommand{\Accept}{\textcolor{black}{\mathsf{Accept}}}
\newcommand{\Protocol}{\Pi}
\newcommand{\TesterFunc}{\mathsf{T}}
\newcommand{\tester}[1]{\TesterFunc^{\left(#1\right)}}
\newcommand{\Tester}[2]{\tester{#1}\left(#2\right)}
\newcommand{\prover}{\mathsf{P}}
\newcommand{\Prover}[1]{\prover^{#1}}
\newcommand{\chProver}{{\widetilde{\prover}}}
\newcommand{\ChProver}[1]{\chProver^{#1}}
\newcommand{\ProofSystem}[2]{\Pi\left(\Prover{#1},\tester{#1}; #2\right)}
\newcommand{\ChProofSystem}[2]{\Pi\left(\ChProver{#1},\tester{#1}; #2\right)}

\newcommand{\outputs}[1]{\text{out}\left[#1\right]}
% \newcommand{\ChProtocolExpanded}{\Protocol(\ChProver, \Verifier)}
% \newcommand{\ProtocolExpanded}{\Protocol(\Prover, \Verifier)}
% \newcommand{\outProtocol}[1]{\Verifier(#1)}

% Testing Prelims
\newcommand{\Proximity}{\tau}
\newcommand{\PropSupportSize}[1]{\text{SuppSize}_{\leq#1 }}

% Models and Testers
\newcommand{\Cond}{\texttt{Cond}}
\newcommand{\PCond}{\texttt{PCond}}
\newcommand{\ICond}{\texttt{ICond}}
\newcommand{\Samp}{\texttt{Samp}}
\newcommand{\BPPTester}{\mathsf{BPP}}
\newcommand{\AMTester}{\mathsf{AM}}
\newcommand{\IPTester}{\mathsf{IP}}
\newcommand{\MATester}{\mathsf{MA}}
\newcommand{\NPTester}{\mathsf{NP}}

\newcommand{\NumSamples}{m}
\newcommand{\yStar}{{y^\star}}
\newcommand{\yStarJ}{{y_{j}^\star}}
\newcommand{\jStar}{{j^\star}}
\newcommand{\Confidence}{\beta}

\newcommand{\PromiseYes}[1]{\Pi^{(\text{Yes}, \alpha)}_{#1}}
\newcommand{\PromiseNo}[1]{\Pi^{(\text{No}, \beta)}_{#1}}
\newcommand{\Promise}[1]{\Pi_{#1}}

\newcommand{\JStar}{j^\star}

\newcommand{\NeighbourhoodExpanded}[2]{U^{\Dist}_{#2}\left(#1\right)}
\newcommand{\Neighbourhood}[1]{\NeighbourhoodExpanded{#1}{\Proximity}}
\newcommand{\NeighbourhoodAlpha}[1]{\NeighbourhoodExpanded{#1}{\alpha}}
\newcommand{\cmark}{\checkmark} % tick
\newcommand{\xmark}{\texttimes} % cross


